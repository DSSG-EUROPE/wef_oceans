\section{Introduction}
Ocean fisheries provide a vital source of food, employment, recreation, trade, and economic security for people throughout the world. Over three billion people depend on marine and coastal resources for their livelihoods, with a global market value of \$3 trillion per year or about 5\% of global GDP. Oceans serve as the world's largest source of protein, with more than 3 billion people depending on the oceans as their primary source of protein~\cite{sgd14}. These resources need to be properly managed if their contribution to the nutritional, economic, and social well-being of the world's growing population is to be sustained. 

\subsection{Illegal, unreported, and unregulated fishing}
Since the advent of industrialised fishing practices in the 1950s fish stocks have been in decline. Large predator species such as tuna are estimated to have declined by 90\%~\cite{Myers2003}. Their depletion not only threatens the future of these fish, but the livelihood of the fishers that depend on them, and the equilibirum of the ocean ecosystem. The proximate causes of overfishing are: limited or ineffective harvest regulations, overcapacity of fishing fleets, destructive fishing practices, and \gls{iuu} fishing. The ultimate causes of overfishing are institutional constraints, and a limited capacity to manage fisheries, including limited information and ability to control illegal activities.

Estimating the extent of the \gls{iuu} fishing problem is difficult, since by nature these fishing practices are illicit and concealed. The \gls{iuu} catch is in addition to a world annual catch of fish and other marine fauna of approximately 80 million tonnes. This \gls{iuu} fishing problem is particularly problematic because it can lead to overfishing for a given maritime region. According to the \gls{fao} Fisheries and Aquaculture Department, illegal fishing causes losses estimated at \$23 billion per year with about 30 percent of illegal fishing in the world occurring in Indonesia. 

The driving force behind \gls{iuu} fishing are similar to those behind many other types of international environmental crime: pirate fishers have a strong economic incentive. Many species of fish, particularly those that have been overexploited and are thus in short supply, are of high financial value. These \gls{iuu} fishing practices are systemic in many fisheries worldwide, and are generally linked to weak governance. \gls{iuu} fishing is a key challenge to overcome in order to achieve sustainably managed fisheries.

\subsection{Sustainable Development Goals}
The \gls{un} coordinates international collaboration in addressing global challenges in environmental protection, whilst fostering social and economic development. The \gls{un} and it's member states devised a set of 17 \gls{sdg} covering a broad range of sustainable development issues. Goal number 14 Life Below Water is concerned with the conservation and sustainable use of the oceans, seas and marine resources. By 2020 this goal sets out to:
\begin{itemize}
    \item effectively regulate fisheries harvesting,
    \item implement science-based management plans, and,
    \item end overfishing, \glsdisp{illegal}{illegal} \glsdisp{unreported}{unreported} and \glsdisp{unregulated}{unregulated} (\gls{iuu}) fishing.
\end{itemize}
If these goals are met then fish stocks can be maintained at the level capable of producing the maximum sustainable long-term yield.

\subsection{Objective}
In collaboration with the \gls{wef} this project set out to create a tool to use data, and data science approaches, to guide and enhance fisheries power to enforce and control \gls{iuu} fishing. The premise of which was to create a vessel fishing risk index based on a series of quantitative metrics to indicate vessels that are likely to engage in \gls{iuu} fishing. A proof-of-concept open source tool was developed using aggregate data sets from sources such as satellite imagery, \gls{sar}, and \gls{ais} data. The tool provides indicators of vessels fishing illegally, allowing for more targeted enforcement whilst encouraging responsible fishing practices.

%\subsection{Challenges}
%There are many challenges associated with detecting IUU fishing. Firstly, the world's ocean comprise the majority of the Earth's surface (71\%). This is a large area to inspect and hence the volumes of data involved are large. There are many vessels in this large area relating to commercial, leisure, or fishing activities. Systems to detect and track these vessels such as \gls{ais} tend to be implemented nationally in \gls{vms}, hence there is little standardization of the data format. Satellite data on the other hand is relatively infrequent, can be obscured by cloud cover, and may not have good coverage in the ocean. Combining these data sources can be difficult to find appropriate images and AIS data. There is also a distinct lack of high-resolution open-source data sources in this domain, due to the cost of data collection.
%
%\subsection{Desired Outcomes}
%A socially desirable outcome for this project would be to successfully demonstrate how these data can be used to identify IUU fishing. In partnership with the WEF and organisations this can be conveyed to policy and decision makers to expand the study. A socially desirable outcome would be to improve detection of illegal fishing via these data sources, a secondary outcome of this would be to provide improved enforcement of illegal fishing, which in turn would improve regulation as it becomes harder to evade capture. The result of this would be to promote more sustainable fishing practice and environmental conservation.
