\section{Previous work}

Ship detection is a relevant enforcement tool, and also can serve as a way understand a myriad of behaviors in the ocean, such as piracy, illegal maritime trafic, and UII fishing. A complete set of literature have used different ship detection algorithms and different sources of geospatial data. For instance, \citeA{Tello2005} uses satellite-based synthetic aperture radar (SAR) and a discrete wavelet transform to capture spectral signals of vessels in the ocean. The use of SAR data to this task is common in the literature \cite{Margarit2009, Brusch2011, Corbane2008, Paes2010}.\footnote{Most of these documents rely on the \textit{TerraSAT} proyect data which have a resolution up to \approx$ 16 meters.} The ability of the SAR sensors to capture signals even in the presence of higher percentage of cloud coverage, give radar data an advantage over visible (optical) data. Nonetheless, vessel detection with SAR data has limitations for identifying small ships, limited coverage, and its automatic interpretation can be a cumbersome process \cite{Zhang2006}.  

\citeA{Corbane2010}, shows a first apporach to vessel detection using panchromatic high-resolution imagery from the SPOT-5 program. The authors used a wavelet transform, which decompose the image according to light frecquency profiles and used a novel preprocessing approach that allows having rapid classifications compared with other algorithms. \citeA{Lebona2016} also uses optical data to track vessels, although the authors use the NASA-VIIRS nightlight data. The results of the classification algortihm are cross-validated using vessel positional AIS data, confirming the correct indetification of 5000-6000 ships per night, but is not clear if this approach is suitable to identify small targets, like fishing vessels.\footnote{VIIRS day-night band, as its predecesor, the NOAA-OLS, has a lower resolution (1 $km^{2}$ at the equator), and other problems like overglooming in the coastlines that can yield false positives. To read a comprehensive assessment of the use of nightlight data to data analysis see \citeA{Min2015}}.

Image classification is not the only approach to track vessels on the ocean. Positional messages are also used to classify ship activity 


