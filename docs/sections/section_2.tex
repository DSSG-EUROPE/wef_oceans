\section{Ocean data}
An effective approach to combatting \gls{iuu} fishing will be multifaceted and require multiple. This approach will most likely have to target the underlying causes of \gls{iuu} fishing, improve traceability within the supply chain, and improve detection and enforcement of \gls{iuu} fishing. Monitoring fishing activity at sea is especially challenging in the high seas.

Technological advancements are providing improved transparency, and traceability in the fish supply chain such as using distributed ledger blockchain technologies. Satellites are also being used to collect ever increasing amounts of vessel tracking data, oceanographic data, and imagery. We are also witnessing the beginnings of autonomous drone usage for monitoring and enforcement of \gls{iuu} fishing.

\subsection{Automatic identification system}
The most comprehensive existing vessel location and tracking service is provided by \glsfirst{ais}. This is an automated, autonomous tracking system, extensively used in the maritime world for the exchange of navigational informational between AIS-equipped terminals. The system was devised for the purposes of collision avoidance and navigation. Transponders installed on vessels broadcast messages that can be received by nearby vessels as well as low-earth orbiting satellites, or land based stations if close to land. 

A vessel's onboard \gls{ais} system consists of one \gls{vhf} transmitter, two \gls{vhf} receivers, and one \gls{vhf} digital selective calling module to receive distress signals. This information is displayed on a standard marine electronic communications link to shipboard display and sensor systems. Positional information is derived from an inbuilt \gls{gps} receiver or an external one. 

These transponders broadcast two types of messages. First, they transmit positional signals every $\approx$ 10 seconds, reporting GPS coordinates and navigational features of the vessel (i.e., speed, course, and heading). The second type of message is static and is reported every $\approx$ 30 seconds, these features include ship name, callsign, length, and port destination. 

Beacons near the shores capture these messages, working as an avoiding collision system with the sea coast and other vessels. This makes AIS an important component of maritime security \cite{Tetreault2005}. Coastal receivers do not always capture AIS signals away from shore, with a maximum range of approximately 80 kilometers range. Low-orbit satellite captured AIS signals collect vessel information further out at sea. This allows us to retrieve vessel tracking data from all over the world, and not only from vessels near seashores.

There are some caveats to using \gls{ais} data. First, \gls{ais} signals fetched by satellites can \textit{bunch together} leading to data loss, since the receiver is only able to process a limited number of signals at a time. Second, the positional accuracy of the signal is not always certain. Hence there is a possible measurement error in the location and other features of the vessel. Lastly, there is the possibility of tampering with the \gls{ais} transponder to for example spoof the vessels involved in illegal activities. The most significant downside of \gls{ais} is that vessels may simply turn off their transponder in order to engage in \gls{iuu} activities.

\subsection{Satellite imagery}
To visually validate the AIS data, we used different sources of proprietary satellite imagery for the same timeframe of the positional AIS data. We rely on two sources of high-definition satellite imagery. The first source is Digital Globe (henceforth GBDX) which reports images with visual and Near Infrared (NIR) bands with a resolution of $\approx$ 3 meters per pixel. Second, we used imagery from Planet Labs, which has higher revisit times that GBDX, but at the cost of lower resolution (from $\approx$ 10 m. to $\approx$ 3 m. over the equator). 

Images from both sources were orthorectified and projected into a common grid. The first process corrects possible terrain distortions due to higher elevation angles \footnote{This angle is commonly known as Nadir. The terrain displacement occurs when satellite sensors are not orthogonal to the earth terrain. Thus the optimal Nadir angle is \ang{90}.} The second process projects the images into a common grid to georeference them into a common grid, usually under a \textit{UTM} local projection. Despite the availability of more advanced image processing tasks, as \textit{Pansharpening} or \textit{Atmospheric compensation}, we decided to use a single approach. Also, some advanced image processing tasks merge visual and NIR bands, hindering the use of a multispectral or thermal analysis. 

\subsection{Synthetic-aperture radar}
