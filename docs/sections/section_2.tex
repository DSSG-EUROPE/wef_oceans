\section{Previous work}
The first step in investigating a myriad of vessel activities including piracy, illegal maritime traffic, and IUU fishing is to find where the ships are. In the literature approaches for ship detection are described using multiple algorithms and applied to various geospatial data sources. For example, \citeA{Tello2005} uses satellite-based synthetic aperture radar (SAR) and a discrete wavelet transform to capture spectral signals of vessels in the ocean. The use of SAR data fort this task is common in the literature \cite{Margarit2009, Brusch2011, Corbane2008, Paes2010} \footnote{Most of these documents rely on the \textit{TerraSAT} project data which have a resolution up to $\approx$ 16 meters.}. The ability of the SAR sensors to capture signals even in the presence of high cloud coverage gives radar data an advantage over visible (optical) data. Nonetheless, vessel detection with SAR data has limitations for identifying small ships, limited coverage, and its automatic interpretation can be a cumbersome process \cite{Zhang2006}.  

\citeA{Corbane2010}, shows a first apporach to vessel detection using panchromatic high-resolution imagery from the SPOT-5 program. The authors used a wavelet transform, which decomposes the image according to light frecquency profiles and used a novel preprocessing approach that allows having rapid classifications compared with other algorithms. \citeA{Lebona2016} also uses optical data to track vessels, although the authors use the NASA-VIIRS nightlight data. The results of the classification algortihm are cross-validated using vessel positional AIS data, confirming the correct indetification of 5000-6000 ships per night, but is not clear if this approach is suitable to identify small targets, like fishing vessels.\footnote{VIIRS day-night band, as its predecesor, the NOAA-OLS, has a lower resolution (1 $km^{2}$ at the equator), and other problems like overglooming in the coastlines that can yield false positives. To read a comprehensive assessment of the use of nightlight data to data analysis see \citeA{Min2015}}.

Image classification is not the only approach to track vessels on the ocean. Positional messages are also used to classify ship activity. The project Global Fishing Watch for example used a labelled set of fishing vessels to classify vessels in automatic identification system (AIS) data as fishing or non-fishing, this is described in \citeA{de2016correction}. This approach relies on a reliable AIS signal, and vessels fishing illegally may not have a transponder, may switch it off, or may spoof its signal to engage in these behaviours. As such this should not be relied upon as the only approach to identify vessels. Here we have used a similar model to identify fishing vessels, but plan to incorporoate other vessel detection methods.
