\section{Data sources}
\subsection{Automatic identification system (AIS)}
Vessel location for the period from May 2016 to June 2017  was captured using the Automatic 
Identification System (AIS) transponders installed in every vessel with a length greater or equal to 
30 meters. These transponders broadcast two types of messages. First, they transmit positional signals 
every 10 seconds, reporting GPS coordinates and navigational features of the vessel (i.e., speed, course, 
and heading). The second type of message is static and reports constant features of the ship, such as name, 
callsign, length, and destination. 

Beacons near the shores capture these messages, working as an avoiding collision system with the sea coast and 
other vessels. This makes AIS an important component of maritime security \cite{Tetreault2005}. 
Coastal receivers do not always capture AIS signals away from shore.\footnote{According to Spire, the coastal 
receivers can only capture data in an 80 kilometers range.} Nonetheless, we use satellite captured AIS signal 
provided by Spire; this allows us to retrieve vessel tracking data from all over the world, and not only from 
vessels near seashores.Some caveats arise with the use of this source. First, AIS signals fetched by satellites
can \textit{bunch together} bunch together leading to data loss, since the satellite is only able to process a 
limited number of signals at a time. Second, accuracy is not always certain. Hence there is a possible measurement 
error in the location and other features of the vessels. Lastly, there is the possibility of AIS tampering in 
vessels involved in illegal activities, such as UUI. 

\subsection{Satellite imagery}

To account for a visual validation of the AIS data, we used different sources of proprietary 
satellite imagery for the same timeframe of the positional AIS data. We rely on two sources of 
high-definition satellite imagery. The first source is Digital Globe (henceforth GBDX) which reports
images with visual and Near Infrared (NIR) bands with a resolution of $\approx$ 3 meters per pixel.
Second, we used imagery from Planet, which has higher revisit times that GBDX, but at the cost of lower 
resolution (from $\approx$ 10 m. to $\approx$ 3 m. over the equator). 

Images from both sources were orthorectified and projected into a common grid. The first process corrects
possible terrain distortions due to higher elevation angles.\footnote{This angle is commonly known as Nadir.
The terrain displacement occurs when satellite sensors are not orthogonal to the earth terrain. Thus the optimal
Nadir angle is \ang{90}.} The second process projects the images into a common grid to georeference them into 
a common grid, usually under a \textit{UTM} local projection. Despite the availability of more advanced image 
processing tasks, as \textit{Pansharpening} or \textit{Atmospheric compensation}, we decided to use a single approach. 
Also, some advanced image processing tasks merge visual and NIR bands, hindering the use of a multispectral 
or thermal analysis. 


