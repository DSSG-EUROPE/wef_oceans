\doublespacing
\noindent The lack of traceability within the fishing industry limits responsible governance, and management of the oceans. As a consequence of this lack of traceability illegal, unregulated and unreported fishing is rife, affecting fisheries worldwide. A major challenge within fish traceability is determining whether fish have been caught sustainably from vessels acting in a legal, and responsible manner. The proof-of-concept system proposed here creates a vessel risk framework to assess the likelihood that a vessel has engaged in illegal, unregulated, or unreported fishing.\\~\\
\noindent Using historical vessel tracking data the system first predicts the likelihood at each time point that a vessel was fishing, using features including its movement and distance from shore. Vessels that are fishing are then scored using multiple indicators to evaluate the risk of these behaviours. Indicators include the likelihood that a vessel has previously fished in a marine protected area or exclusive economic zone, and the intermittency of the vessels automatic identification system positional signal. This information is displayed in web application that allows the user to weight the components according to the use case, or for convenience combined into a unified vessel risk score.\\~\\
\noindent This proposed fishing risk framework combines multiple data sets by correlating these tracking data with satellite imagery. For all historical vessel tracking and all time points the available satellite imagery is found, this could be used as further evidence to substantiate our risk indicator. This proof-of-concept shows show how multiple data sources can be combined to start building a library of historic evidence and data of a vessels behaviour. This gives governments, retailers, coastguards and enforcement agencies the information they need to improve traceability within the fishing industry, better manage the oceans, and conduct more effective enforcement.
